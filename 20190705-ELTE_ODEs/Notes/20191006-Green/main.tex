\documentclass[12pt,a4paper]{article}

\usepackage[utf8]{inputenc}
\usepackage[english]{babel}
\usepackage{amssymb, amsmath}
\usepackage{fullpage}
\usepackage{parskip}
\usepackage{url}
\usepackage{color}
\usepackage{enumitem}

\newcommand{\from}{\colon}
\newcommand{\IR}{\mathbb{R}}
\newcommand{\IN}{\mathbb{N}}
\newcommand{\norm}[1]{\left\|#1\right\|}

\begin{document}
    
    %%%%%%%%%%%%%%%%%%%%%%%%%%%%%%%%%%%%%%%%%%%%%%%%%%
    Differential equations: boundary value problems
    \hfill
    ELTE, autumn 2019/20
    %%%%%%%%%%%%%%%%%%%%%%%%%%%%%%%%%%%%%%%%%%%%%%%%%%
    
    %%%%%%%%%%%%%%%%%%%%%%%%%%%%%%%%%%%%%%%%%%%%%%%%%%
    \subsection*{Boundary value problems (DRAFT \today)}
    %%%%%%%%%%%%%%%%%%%%%%%%%%%%%%%%%%%%%%%%%%%%%%%%%%
    
    %%%%%%%%%%%%%%%%%%%%%%%%%%%%%%%%%%%%%%%%%%%%%%%%%%
    \subsubsection*{Introduction}
    %%%%%%%%%%%%%%%%%%%%%%%%%%%%%%%%%%%%%%%%%%%%%%%%%%
    
    
    Let $I = (a, b)$ be a non-trivial bounded interval.
    %
    An important class of differential equations
    (historically, theoretically, didactically and practically)
    are
    of the form
    %
    \begin{align}
        \label{e:u}
        %
        -(p u')' + q u = f
        \qquad\text{on}\qquad
        I
    \end{align}
    %
    where 
    $u \from I \to \IR$
    is the unknown function,
    while $p$, $q$ and $g$ are known, sufficiently smooth functions.
    %
    Typically, 
    \begin{align}
        \label{e:p0}
        \text{$p \geq 0$ is non-negative.}
    \end{align}

    %%
    
    \emph{Example}.
    %
    Consider $u'' + a_1 x' + a_0 x = g$.
    %
    Set $p(s) := \exp(\int_a^s a_1(\tau) d\tau)$
    and $q := -p a_0$.
    
    
    %%
    
    To streamline the notation we introduce 
    the differential operator $L$,
    \begin{align}
        \label{e:L}
        L u :=
        -(p u')' + q u
        .
    \end{align}
    %
    Thus, Eqn.~\eqref{e:u} becomes
    \begin{align}
        \label{e:Lu}
        L u = f
        .
    \end{align}

    
    %%
    
    The equation \eqref{e:u} is linear in $u$ and involves two derivatives.
    %
    %
    From the ODE theory we expect that
    the homogeneous equation
    \begin{align}
        \label{e:L0}
        L \varphi = 0
    \end{align}
    has two (linearly independent) solutions
    $\varphi_1$ and $\varphi_2$,
    forming a so-called fundamental system.
    %
    %
    If, moreover, $\varphi_0$ is a particular solution
    to the original non-homogeneous equation \eqref{e:Lu}
    then so is any combination,
    for any $c_1, c_2 \in \IR$,
    \begin{align}
        L(\varphi_0 + c_1 \varphi_1 + c_2 \varphi_2)
        =
        L \varphi_0
        +
        c_1 L \varphi_1
        +
        c_2 L \varphi_2
        = 
        g
        .
    \end{align}
    %
    %
    The solution space is two-dimensional,
    parameterized by the two constants $c_1, c_2 \in \IR$.
    %
    %
    To fully specify the solution,
    two additional constraints are needed.
    
    
    
    %%%%%%%%%%%%%%%%%%%%%%%%%%%%%%%%%%%%%%%%%%%%%%%%%%
    \subsubsection*{The operator $L$ is self-adjoint (almost)}
    %%%%%%%%%%%%%%%%%%%%%%%%%%%%%%%%%%%%%%%%%%%%%%%%%%
    
    The particular form of $L$ is significant 
    for the following reason,
    called ``Lagrange identity'' in a more general context.
    
    \emph{Observation}.
    %
    For any two smooth functions $u$ and $v$,
    \begin{align}
        \label{e:LI}
        %
        u L v - v L u
        =
        - u (p v')'
        + v (p u')'
        =
        (p (u' v - u v'))'
        .
    \end{align}

    %
    
    The significance of the observation 
    is further seen by defining
    the bilinear form
    \begin{align}
        B(u, v) 
        :=
        \int_I (L u)(x) v(x) dx
        .
    \end{align}
    %
    %
    Flipping $L$ onto $v$ using \eqref{e:LI}, 
    we have the identity
    \begin{align}
        B(u, v)
        =
        B(v, u)
        +
        \int_I (p (u' v - u v'))' dx
        ,
    \end{align}
    which says that $B$ is almost symmetric
    but for the last anti-symmetric term.
    %
    %
    However,
    if we restrict $u$ and $v$ 
    to the \emph{vector space} of functions
    that vanish on the boundary of $I$
    (for example),
    the bilinear form \emph{is} symmetric.
    %
    %
    This unleashes operator-theoretic tools
    that form the basis of 
    the \emph{finite element method}
    for boundary value problems like \eqref{e:Lu}.
    
    
    %%%%%%%%%%%%%%%%%%%%%%%%%%%%%%%%%%%%%%%%%%%%%%%%%%
    \subsubsection*{Boundary conditions I}
    %%%%%%%%%%%%%%%%%%%%%%%%%%%%%%%%%%%%%%%%%%%%%%%%%%
    
    %
    
    In \emph{initial value problems},
    we specify two conditions
    on one end of the interval, 
    such as
    $
        u(a) := u_0
    $
    and
    $
        u'(a) := u_1
    $.
    %
    %
    Such equations typically model oscillatory evolution over time.
    
    %
    
    In \emph{boundary value problems}, 
    we specify conditions at both ends of the interval,
    called boundary conditions.
    %
    For example,
    \begin{align}
        \text{homogeneous Dirichlet condition at $a$}:
        & 
        \qquad
        D_a u := u(a) \stackrel{!}{=} 0
        %
        \\
        %
        \text{homogeneous Neumann condition at $b$}:
        &
        \qquad
        N_b u := u'(b) \stackrel{!}{=} 0
        .
    \end{align}
    
    %
    
    More generally,
    so-called
    Robin boundary conditions can be imposed:
    \begin{subequations}
        \label{e:H}
        \begin{align}
            H_a u := \alpha_0 u(a) + \alpha_1 u'(a) & \stackrel{!}{=} \eta_a
            \\
            H_b u := \beta_0 u(b) + \beta_1 u'(b) & \stackrel{!}{=} \eta_b
        \end{align}
    \end{subequations}
    where 
    $\alpha_i$, $\beta_i$ and $\eta_x$ are constants.
    %
    Note that $H_a$ and $H_b$ are
    \emph{linear operators}
    that take a function 
    and return a real number (if defined).
    %
    In more general boundary value problems
    posed on $d$-dimensional domains,
    the boundary operator
    returns a function (called \emph{trace} of $u$) 
    on the $(d - 1)$-dimensional boundary.
    %
    
    For the candidate solution in the form
    $u = \varphi_0 + c_1 \varphi_1 + c_2 \varphi_2$
    we can write the set of 
    boundary conditions in matrix-vector form,
    %
    \begin{align}
        \begin{pmatrix}
            H_a \varphi_1 & H_a \varphi_2
            \\
            H_b \varphi_1 & H_b \varphi_2
        \end{pmatrix}
        \begin{pmatrix}
            c_1 \\ c_2
        \end{pmatrix}
        =
        \begin{pmatrix}
            \eta_a \\ \eta_b
        \end{pmatrix}
        -
        \begin{pmatrix}
            H_a \varphi_0 \\ H_b \varphi_0
        \end{pmatrix}
        ,
    \end{align}
    %
    providing conditions 
    on the unknonwn coefficients $c_1$ and $c_2$.
    %
    %
    If the determinant of the matrix is non-zero,
    these coefficients are uniquely determined
    (but otherwise there may be no solution, one solution or infinitely many solutions).
    
    %
    
    \emph{Example}.
    %
    Take 
    $p := 1$ and $q := -(\pi k)^2$,
    where $k \in \IN$,
    and $g = 0$,
    on the interval $(0, 1)$
    subject to 
    the homogeneous Dirichlet boundary conditions
    $u(0) = 0$
    and
    $u(1) = 0$.
    %
    %
    There is no need for the particular solution,
    in other words $\varphi_0 = 0$.
    %
    Determine $c_1$ and $c_2$.
    
    %
    
    
    %%%%%%%%%%%%%%%%%%%%%%%%%%%%%%%%%%%%%%%%%%%%%%%%%%
    \subsubsection*{Boundary conditions II}
    %%%%%%%%%%%%%%%%%%%%%%%%%%%%%%%%%%%%%%%%%%%%%%%%%%
    
    In practice,
    that is in algorithms and software packages
    for boundary value problems,
    there are other ways to impose 
    the (linear) boundary conditions \eqref{e:H}.
    %
    %
    A relatively easy way is to 
    decompose $u = \tilde{u} + u_0$
    where
    $u_0$ is
    \emph{any} function that satisfies
    the boundary conditions \eqref{e:H}.
    %
    %
    Then,
    setting 
    $\tilde{f} := f - L u_0$,
    the function $\tilde{u}$
    satisfies
    the non-homogeneous boundary value problem
    \begin{align}
        L \tilde{u} = \tilde{f}
    \end{align}
    with homogeneous boundary conditions
    $H_{\cdot} \tilde{u} = 0$.
    
    %
    
    \emph{Example}.
    %
    On $I = (-1, 1)$,
    take
    \begin{align}
        L u := -u'' + u = 0
        \quad\text{with}\quad
        u(\pm 1) = \pm 1
        .
    \end{align}
    %
    Then $u_0(x) := x$ satisfies the boundary conditions.
    %
    Therefore, if $\tilde{u}$ solves
    \begin{align}
        (L \tilde{u})(x) \stackrel{!}{=} -(L u_0)(x) = x
        \quad
        \forall x \in (-1, 1)
        \quad\text{with}\quad
        \tilde{u}(\pm 1) = 0,
    \end{align}
    then $u := \tilde{u} + u_0$ solves the original problem.
    
    %
    
    In this way,
    we do not need 
    the fundamental system $\{ \varphi_1, \varphi_2 \}$,
    which is in general hard to find anyway.
    %
    %
    In the following we therefore focus on 
    homogeneous boundary conditions.
    
    
    %%%%%%%%%%%%%%%%%%%%%%%%%%%%%%%%%%%%%%%%%%%%%%%%%%
    \subsubsection*{Green's function}
    %%%%%%%%%%%%%%%%%%%%%%%%%%%%%%%%%%%%%%%%%%%%%%%%%%
    
    We consider boundary value problems 
    $L u = f$ \eqref{e:Lu}
    on an interval $I$
    with homogeneous boundary conditions.
    %
    %
    The Green's function is largely a theoretical device
    for representing solutions 
    that is based on 
    the following idea
    (that also works in higher dimensions).
    
    %
    
    Let $\delta$ denote the Dirac functional at $0$
    that formally verifies 
    \begin{align}
        \label{e:delta}
        %
        \int_I f(y) \delta(y - x) dy
        =
        f(x)
        \quad
    \end{align}
    for all smooth functions $f$.
    
    %
    
    \emph{Example}.
    %
    Let $H$ be the Heaviside function
    with $H(x) = 0$ for $x < 0$
    and $H(x) = 1$ for $x > 0$.
    %
    Show that
    \begin{align}
        \int_{\IR} f(y) H'(y - x) dy
        =
        f(x)
        \quad
        \forall f \in C^1(\IR)
        .
    \end{align}
    
    %
    
    The Green's function to the operator $L$ 
    is a scalar-valued function 
    $G \from I \times I \to \IR$
    of two variables
    such that
    \begin{align}
        \label{e:G}
        %
        L [ G(\cdot, y) ](x)
        =
        \delta(y - x)
        %
        \quad
        \forall x \in I
        ,
    \end{align}
    where we agree that $L$ acts on the first variable of $G$.
    %
    %
    Now set
    \begin{align}
        \label{e:uG}
        %
        u(x) := \int_I G(x, y) f(y) dy
        .
    \end{align}
    %
    %
    Then, bravely exchanging differentiation in $x$ and integration in $y$,
    \begin{align}
        (L u)(x)
        \stackrel{\eqref{e:uG}}{=}
        \int_I L[ G(\cdot, y) ](x) f(y) dy
        \stackrel{\eqref{e:G}}{=}
        \int_I \delta(y - x) f(y) dy
        \stackrel{\eqref{e:delta}}{=}
        f(x)
        %
        \quad
        \forall x \in I
        .
    \end{align}

    %
    
    \emph{Remark}.
    %
    If $x \mapsto G(x, y)$ satisfies
    the homogeneous boundary conditions
    then so does \eqref{e:uG}.
    
    %
    
    %%%%%%%%%%%%%%%%%%%%%%%%%%%%%%%%%%%%%%%%%%%%%%%%%%
    \subsubsection*{Green's function for $L u = -(p u')' + q u$}
    %%%%%%%%%%%%%%%%%%%%%%%%%%%%%%%%%%%%%%%%%%%%%%%%%%
    
    For the one-dimensional problem \eqref{e:Lu}
    with homogeneous boundary conditions
    a Green's function can be constructed
    from the fundamental system $\{ \varphi_1, \varphi_2 \}$.
    %
    %
    Recall that this means in particular that $L \varphi_i = 0$.
    
    %
    
    First, note that
    the Lagrange identity \eqref{e:LI}
    implies that 
    \begin{align}
        \label{e:c}
        %
        c := (\varphi_1 \varphi_2' - \varphi_1' \varphi_2) p
        \quad
        \text{is constant}.
    \end{align}
    %
    %
    We assume that this constant is non-zero.

    %
    
    Now set
    \begin{align}
        G(x, y)
        :=
        \begin{cases}
            \frac1c \varphi_1(x) \varphi_2(y) & \text{if} \quad x \geq y
            \\
            \frac1c \varphi_1(y) \varphi_2(x) & \text{if} \quad y \geq x
            .
        \end{cases}
    \end{align}
    
    %
    
%     \emph{Remark}.
%     %
%     We can write $G$ using the Heaviside function as
%     \begin{align}
%         G(x, y)
%         =
%         \varphi_1(x) \varphi_2(y) H(x - y)
%         +
%         \varphi_1(y) \varphi_2(x) H(y - x)
%         .
%     \end{align}

    %
    
    \emph{Remark}.
    The function $u$ defined by \eqref{e:uG}
    satisfies the \emph{homogeneous} boundary conditions
    because
    $G(\cdot, y)$ does (for any $y \in I$).
    
    %
    
    \emph{Remark}.
    The function $x \mapsto G(x, y)$ is 
    continuous but likely not differentiable at $x = y$.
    %
    We write $G'$ to mean the derivative 
    with respect to the first variable.
    
    %
    
    Now we verify that $G$ is indeed a Green's function 
    for $L$ by checking \eqref{e:G}.
    %
    %
    Specifically, we check that
    \begin{align}
        \label{e:?}
        %
        \lim_{\epsilon \searrow 0}
        \int_{y - \epsilon}^{y + \epsilon}
        L [G(\cdot, y)](x) dx
        =
        1
        \qquad
        \forall y \in I
        .
    \end{align}
    %
    %
    Since the integrand is zero whenever $x \neq y$,
    this implies that it equals $\delta(x - y)$,
    which is equivalent to \eqref{e:G}.

    Indeed,
    using the Lagrange identity \eqref{e:LI}
    with $u = G(\cdot, y)$ and $v = 1$,
    \begin{align}
        \text{LHS}\eqref{e:?}
        & =
        -
        \int_{y - \epsilon}^{y + \epsilon}
        (p G'(\cdot, y))'(x) 
        dx
        \\
        & =
        \left.
        - p G'(x, y)
        \right|_{x = y - \epsilon}^{x = y + \epsilon}
        \\
        & =
        -
        \frac1c
        (
            p(y + \epsilon) \varphi_1'(y + \epsilon) \varphi_2(y)
            -
            p(y - \epsilon) \varphi_2'(y - \epsilon) \varphi_1(y)
        )
        \\
        & \to 
        \frac1c (\varphi_1 \varphi_2' - \varphi_1' \varphi_2)(y) p(y)
        \quad \text{as} \quad
        \epsilon \searrow 0
        \\
        & = 1
        \quad \text{by} \quad \eqref{e:c}
        .
    \end{align}

    \emph{Remark}.
    We assumed that $p$ and $\varphi_i'$ are continuous.
    
    %
    
    \emph{Example}. {\color{red} TODO: hat/tent} 
    
    \vfill\hfill 
    R.A., \today
\end{document}
