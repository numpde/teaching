\documentclass[12pt,a4paper]{article}

\setcounter{section}{1}

\usepackage[utf8]{inputenc}
\usepackage[english]{babel}
\usepackage{amssymb, amsmath}
\usepackage{fullpage}
\usepackage{parskip}
\usepackage{url}
\usepackage{enumitem}

\newcommand{\from}{\colon}
\newcommand{\norm}[1]{\left\|#1\right\|}

\begin{document}
    Differential equations
    \hfill
    ELTE, autumn 2019/20
    
    
    %%%%%%%%%%%%%%%%%%%%%%%%%%%%%%%%%%%%%%%%%%%%%%%%%%
    \subsection{}
    %%%%%%%%%%%%%%%%%%%%%%%%%%%%%%%%%%%%%%%%%%%%%%%%%%
    
    Consider $C^1(\mathbb{R})$ 
    solutions of the differential equation $\dot{x}(t) = -x(t)$.
    \begin{itemize}
    \item 
        Sketch all solutions in the $x$-$t$ plane.
    \item
        What do they have in common?
    \item
        For $T > 0$,
        how do
        $x(0) \mapsto x(T)$
        and
        $x(0) \mapsto x(-T)$
        differ qualitatively?
    \item
        Parameterize all solutions implicitly
        as $F(x, t, \lambda) = 0$.
    \item
        Indicate the ``Lie group orbits'' $\lambda \mapsto F(x, t, \lambda)$ for various fixed $(x, t)$.
    \item
        Find and solve a differential equation
        for the curve in the $x$-$t$ plane
        that goes
        through $(c, 0)$
        and is perpendicular
        to the solutions.
    \end{itemize}
    
    
    %%%%%%%%%%%%%%%%%%%%%%%%%%%%%%%%%%%%%%%%%%%%%%%%%%
    \subsection{}
    %%%%%%%%%%%%%%%%%%%%%%%%%%%%%%%%%%%%%%%%%%%%%%%%%%
    
    Find, if possible,
    a differential equation ``$\dot{x}(t) = \ldots$''
    with a nontrivial solution in $C^1([0, \infty))$
    for which
    $x(t) = 0$ for all $t$ sufficiently large ---
    i.e.~$x$ converges in finite time.
    
    
    
    %%%%%%%%%%%%%%%%%%%%%%%%%%%%%%%%%%%%%%%%%%%%%%%%%%
    \subsection{}
    %%%%%%%%%%%%%%%%%%%%%%%%%%%%%%%%%%%%%%%%%%%%%%%%%%
    
    
    Let $\gamma$, $g$ and $a \geq 0$ be continuous real-valued
    functions on $[0, T]$.
    %
    
    \begin{itemize}
    \item
        Sketch 
        the function
        $
            s \mapsto A(s) :=
            \exp( \int_s^T a(\tau) d\tau )
        $.
    \end{itemize}
    
    Suppose 
    %
    \begin{align}
        \gamma(t) 
        \leq 
        g(t) +
        \int_0^t \gamma(s) a(s) ds
        %
        \quad
        %
        \forall t \in [0, T]
        .
    \end{align}

    
    Prove
    
    \begin{itemize}
    \item 
        the Gronwall--Bellman inequality:
        %
        \begin{align}
            \gamma(T) 
            \leq
            g(T) +
            \int_0^T g(s) a(s) A(s) ds
            .
        \end{align}
    \item
        assuming that $g$ is continuously differentiable,
        the corollary:
        %
        \begin{align}
            \gamma(T) 
            \leq 
            g(0) A(0)
            +
            \int_0^T g'(s) A(s) ds
            .
        \end{align}
    \end{itemize}
        
    It may be helpful to
    start with a differential inequality for 
    $x(t) := \int_0^t \gamma(s) a(s) ds$.

    
    %%%%%%%%%%%%%%%%%%%%%%%%%%%%%%%%%%%%%%%%%%%%%%%%%%
    \subsection{}
    %%%%%%%%%%%%%%%%%%%%%%%%%%%%%%%%%%%%%%%%%%%%%%%%%%
    
    Fix $T > 0$,
    suppose $f \from [0, T] \times \mathbb{R} \to \mathbb{R}$
    is Lipschitz continuous in the second variable.
    %
    Assume there exists a solution to
    the differential equation 
    $\dot{x}(t) = f(t, x(t))$ on the interval $(0, T)$
    for any initial value near some fixed $x_0$.
    
    \begin{itemize}
    \item
        Show that 
        $x(0) \mapsto x(T)$
        is Lipschitz continuous.
        %
        Is your statement sharp?
    \item
        Do we need to \emph{assume} existence?
    \item
        Suppose $f$ depends on a parameter $p$.
        Fix $x_p(0)$.
        What is the derivative of the solution $x_p(T)$ w.r.t.~$p$?
        Propose conditions on $f$ for the derivative to exist.
    \end{itemize}
    
    

    
    %%%%%%%%%%%%%%%%%%%%%%%%%%%%%%%%%%%%%%%%%%%%%%%%%%
    \subsection{}
    %%%%%%%%%%%%%%%%%%%%%%%%%%%%%%%%%%%%%%%%%%%%%%%%%%

    Develop a solution formula to
    the initial value problem
    \begin{align}
        \dot{x}(t) = g(t) h(x(t)),
        \qquad
        x(t_0) = x_0
        ,
    \end{align}
    in terms of the given functions $g$ and $h$.
    %
    Apply to
    \begin{itemize}
    \item
        $\dot{x}(t) = t \, x(t)$,
    \item
        $\dot{x}(t) = t / x(t)$,
    \item
        $\dot{x}(t) = x^2(t)$,
    \item  
        $\dot{x}(t) = \sqrt{ x(t) }$,
    \item
        $\dot{x}(t) = (1 + t^2) (1 + x^2(t))$,
    \item
        $\dot{x}(t) = x(t) \log x(t)$.
    \end{itemize}
    %
    What is the maximal temporal interval of existence in each case?

    
    %%%%%%%%%%%%%%%%%%%%%%%%%%%%%%%%%%%%%%%%%%%%%%%%%%
    \subsection{}
    %%%%%%%%%%%%%%%%%%%%%%%%%%%%%%%%%%%%%%%%%%%%%%%%%%
    
    The following are simple (and simplistic) models for 
    the volume of a growing tumor:
    \begin{itemize}
    \item 
        $\dot{V} = a V$
        \hfill
        (Exponential)
    \item
        $\dot{V} = a V^b$
        \hfill
        (Mendelsohn)
    \item
        $\dot{V} = a V (1 - \frac1b V)$
        \hfill
        (Logistic)
    \item
        $\dot{V} = a V \, (b + V)^{-1}$
        \hfill
        (Linear)
    \item
        $\dot{V} = a V \, (b + V)^{-1/3}$
        \hfill
        (Surface)
    \item
        $\dot{V} = a V \log \frac{b}{c + V}$
        \hfill
        (Gompertz)
    \item
        $\dot{V} = a V^{2/3} - b V$
        \hfill
        (Bertalanffy)
    \end{itemize}
    %
    Herein, $a$, $b$ and $c$ are positive constants.
    %
    For each model, answer the following questions:
    %
    \begin{enumerate}
    \item 
        What is the maximum size of the tumor?
    \item
        What is the doubling time?
    \item
        What is the condition on the constants for positive growth?
    \item
        What a possible motivation for the model?
    \end{enumerate}
    
    %
    
    Now add a term ``$-\gamma V$''.
    %
    \begin{enumerate}[resume]
    \item
        Interpret this term.
    \item
        What is now the maximum size of the tumor?
    \item
        What is the minimum concentration $\gamma$ for complete cure?
    \end{enumerate}
    
    %
    
    Using the experimental data from
    \begin{quote}
        \url{https://www.nature.com/articles/s41598-019-39109-1/figures/4}
    \end{quote}
    select a suitable model for Fig.~4a and 4b
    and estimate the parameters.

    
    %%%%%%%%%%%%%%%%%%%%%%%%%%%%%%%%%%%%%%%%%%%%%%%%%%
    \subsection{}
    %%%%%%%%%%%%%%%%%%%%%%%%%%%%%%%%%%%%%%%%%%%%%%%%%%
    
    Give the maximal interval of existence $(0, t^\star)$
    for the initial value problem
    $\dot{x}(t) = x^2(t)$
    with
    $x(0) := x_0 > 0$.
    
    \begin{itemize}
    \item 
        Why is $t^\star$ called the ``blow up time''?
    \item
        Can you continue the solution past the blow up time?
    \end{itemize}

    
    Now consider 
    $\dot{z}(t) = z^2(t)$, $t > 0$,
    for a complex-valued function $z$
    with the intial value
    $z(0) = x_0 + i \epsilon$,
    where $x_0 > 0$ and $\epsilon \neq 0$ are real.
    %
    Where needed, assume $x_0 = 1$.

    \begin{itemize}
    \item 
        What happens to the blow up time?
    \item
        Decomposing $z = x + i y$,
        write down the differential equations for $x$ and $y$.
    \item
        Argue that $x^2(t) + (y(t) - R_0)^2 = R_0^2$
        for a certain real constant $R_0$
        and
        comment.
    \item
        Sketch the evolution in $(t, x, y)$-space 
        for a small $\epsilon > 0$.
        %
        What happens for $\epsilon \searrow 0$?
    \end{itemize}

    
    
    %%%%%%%%%%%%%%%%%%%%%%%%%%%%%%%%%%%%%%%%%%%%%%%%%%
    \subsection{}
    %%%%%%%%%%%%%%%%%%%%%%%%%%%%%%%%%%%%%%%%%%%%%%%%%%
    
    Develop a solution formula to
    the initial value problem
    \begin{align}
        \dot{x}(t) = a(t) x(t) + b(t)
        ,
        \qquad
        x(t_0) = x_0
        ,
    \end{align}
    in terms of the given functions $a$ and $b$.
    %
    \begin{itemize}
    \item
        Suppose $a$ and $b$ are indenpendent of $t$.
        Investigate the limits $a \to 0$ and $b \to 0$.
    \item
        Apply your formula to $t \, \dot{x}(t) = 2 x(t) + t$.
        %
        Don't forget to discuss $t = 0$.
    \item
        Check your formula on examples of your choice.
    \item
        Is this a good model for global seafood stock?
    \item
        Via a suitable transformation,
        solve 
        $\dot{y}(t) = y(t) - y^\alpha(t)$,
        where $\alpha \in \mathbb{R}$.
        %
        What is remarkable about this 
        ``Bernoulli differential equation''?
    \item
        Solve the ``logistic differential equation''
        $\dot{y}(t) = y(t) (1 - y(t))$
        with $0 \leq y(0) \leq 1$.
        %
        Explain this model
        in terms of ``growth'' and ``competition''.
    \item
        Propose a better model for global seafood stock.
    \end{itemize}

    
    %%%%%%%%%%%%%%%%%%%%%%%%%%%%%%%%%%%%%%%%%%%%%%%%%%
    \subsection{}
    %%%%%%%%%%%%%%%%%%%%%%%%%%%%%%%%%%%%%%%%%%%%%%%%%%
    
    Propose a nontrivial system of differential equations
    ``$\dot{x}(t) = \ldots$,
    $\dot{y}(t) = \ldots$''
    together with an initial condition at $t = 0$
    for which the function
    %
    \begin{align}
        t \mapsto v(t) := x^2(t) + y^2(t)
    \end{align}
    %
    satisfies, for all $t > 0$,
    %
    \begin{itemize}
    \item
        $\dot{v}(t) = -2 v(t)$.
    \item 
        $\dot{v}(t) = 0$.
    \item
        $\dot{v}(t) = 2 x(t) + 2 y(t)$.
    \end{itemize}
    %
    In each case, 
    comment graphically on the possible trajectories $t \mapsto (x(t), y(t))$ in $\mathbb{R}^2$.

    

    %%%%%%%%%%%%%%%%%%%%%%%%%%%%%%%%%%%%%%%%%%%%%%%%%%
    \subsection{}
    %%%%%%%%%%%%%%%%%%%%%%%%%%%%%%%%%%%%%%%%%%%%%%%%%%
    
    On the ``spatial'' interval $I := (0, 1)$
    define
    the functions
    $
        \varphi_k := \sqrt{2} \sin(\pi k \,\cdot\,)
    $.
    %
    Check whether the set
    $\Phi := \{ \varphi_k : k = 1, 2, 3, \ldots \}$
    is
    %
    \begin{itemize}
    \item
        orthonormal
        in the Lebesgue space $L_2(I)$,
    \item
        a basis thereof.
    \end{itemize}

    Now consider the heat equation
    \begin{align}
        \partial_t u(t, x) - \partial_{x x} u(t, x)
        & =
        f(t, x),
        \qquad
        t > 0,
        \quad
        x \in I
        ,
    \end{align}
    with the initial condition $u(0, \,\cdot\,) = g$
    and the boundary conditions $u|_{\partial I} = 0$.
    
    \begin{itemize}
    \item
        What is the physical motivation for this PDE?
    \item
        Solve the PDE formally by expansion in a suitable spatial basis.
    \item
        What qualitative properties of $u$ can you infer?
    \item
        Can you propose a suitable functional framework?
    \item
        Does the solution $u$ depend continuously 
        on the data $f$ and $g$?
    \end{itemize}


    %%%%%%%%%%%%%%%%%%%%%%%%%%%%%%%%%%%%%%%%%%%%%%%%%%
    \subsection{}
    %%%%%%%%%%%%%%%%%%%%%%%%%%%%%%%%%%%%%%%%%%%%%%%%%%
    
    Let $A \in \mathbb{R}^{N \times N}$ be a symmetric positive definite matrix.
    %
    Let $J := (0, T)$ be a nontrivial interval.
    %
    Let $g \in \mathbb{R}^N$ and $f \from J \to \mathbb{R}^N$ continuous.
    %
    Consider the bilinear form
    \begin{align}
        B(u, v) :=
        \int_J \{
            \langle A u, v \rangle
            +
            \langle A^{-1} \dot{u}, \dot{v} \rangle
        \} dt
        +
        \langle u(T), v(T) \rangle
    \end{align}
    and the functional
    \begin{align}
        F(v) :=
        \int_J 
            \langle f, v + A^{-1} \dot{v} \rangle
        dt
        +
        \langle g, v(0) \rangle
        ,
    \end{align}
    defined for
    arbitrary continuously
    differentiable 
    $u, v \from J \to \mathbb{R}^N$.
    
    \begin{itemize}
    \item
        What can you say about the functional
        $
            \mathcal{J}(u) :=
            \frac12 B(u, u) - F(u)
        $?
    \item
        What is the first order condition for the minimum of $\mathcal{J}$?
    \item
        What is the differential equation satisfied by the minimum?
    \item
        What are practical implications?
    \end{itemize}

    

    
    %%%%%%%%%%%%%%%%%%%%%%%%%%%%%%%%%%%%%%%%%%%%%%%%%%
    \subsection{}
    %%%%%%%%%%%%%%%%%%%%%%%%%%%%%%%%%%%%%%%%%%%%%%%%%%
    
    Consider the first order system of differential equations
    for $N$ functions $\mathbf{x}_i \from \mathbb{R} \to \mathbb{R}^d$,
    \begin{align}
        \dot{\mathbf{x}}_i
        =
        \frac{1}{N} 
        \sum_{j \neq i}
        \phi(|\mathbf{x}_j - \mathbf{x}_i|)
        ( \mathbf{x}_j - \mathbf{x}_i )
        ,
        \quad
        i = 1, \ldots, N
        ,
    \end{align}
    where $\phi$ is the ``influence function'',
    say $\phi(r) := (1 + r)^{-s}$ for some $s > 0$.
    
    \begin{itemize}
    \item 
        Explain qualitatively the behavior of the model.
    \item
        What could this be a model for?
    \item
        Find an invariant.
    \item
        What do you expect to happen in the long run?
    \item
        Can you verify your hypothesis?
    \end{itemize}
    
    Consider the modified system 
    \begin{align}
        \dot{\mathbf{x}}_i
        =
        \frac{1}{N}
        \sum_{j \neq i}
        \frac{ \phi_{i j} }{ \sum_k \phi_{i k} }
        ( \mathbf{x}_j - \mathbf{x}_i )
        ,
    \end{align}
    where $\phi_{i j} := \phi(| \mathbf{x}_j - \mathbf{x}_i |)$.
    
    \begin{itemize}
    \item 
        What could be the motivation for this modification?
    \item
        How does this system differ from the previous one in an essential way?
    \end{itemize}

    

    

    

    
    %%%%%%%%%%%%%%%%%%%%%%%%%%%%%%%%%%%%%%%%%%%%%%%%%%
    \subsection{}
    %%%%%%%%%%%%%%%%%%%%%%%%%%%%%%%%%%%%%%%%%%%%%%%%%%
    
    These days, papers are published with a short abstract
    or even with a few ``highlights'' as bullet points. 
    %
    Write the ``highlights''
    emphasizing the role of differential equations for 
    \begin{quote}
        \textsc{R.~M.~Solow},
        \emph{A contribution to the theory of economic growth},
        The quarterly journal of economics, 70 (1956), pp.~65--94.
    \end{quote}
    according to Elsevier's guidelines: 
    \begin{quote}
        {\small\url{https://www.elsevier.com/authors/journal-authors/highlights}}
    \end{quote}
    
    

    %%%%%%%%%%%%%%%%%%%%%%%%%%%%%%%%%%%%%%%%%%%%%%%%%%
    \subsection{}
    %%%%%%%%%%%%%%%%%%%%%%%%%%%%%%%%%%%%%%%%%%%%%%%%%%
    
    The following is from a problem 
    in the IMO 2011
    by Geoffrey Smith (UK):
    %
    \begin{quote}
        Let $S$ be a set of $N \geq 2$ points in the plane (no three points are collinear). 
        A \emph{windmill} 
        is a process that 
        starts with a line $\ell_0$ through a single point $P_0 \in S$. 
        The line rotates clockwise about the pivot $P_0$ until 
        it meets some other point $P_1 \in S$. 
        This point takes over as the new pivot, and so on.
    \end{quote}
    %
    Describe this windmill as a differential equation.
    
    Optionally, solve the problem itself:
    %
    \begin{quote}
        Show that there is a choice of $P_0$ and $\ell_0$
        for which
        the resulting windmill 
        meets each point of $S$ infinitely many times.
    \end{quote}
    
    %
    
    
    %%%%%%%%%%%%%%%%%%%%%%%%%%%%%%%%%%%%%%%%%%%%%%%%%%
    \subsection{}
    %%%%%%%%%%%%%%%%%%%%%%%%%%%%%%%%%%%%%%%%%%%%%%%%%%
    
    The Navier--Stokes equations 
    in cylindrical coordinates
    for a fluid flowing
    through a round pipe
    reduce 
    (under reasonable assumptions: steady, symmetric, laminar and fully developed flow)
    to 
    \begin{align}
        \frac{1}{r}
        \frac{d}{dr} \left( r \mu \frac{d u}{dr} \right)
        =
        \frac{d p}{d z}
        ,
    \end{align}
    where $u$ is the axial velocity along the pipe
    (independent of $z$)
    and $p$ is the pressure
    (independent of $r$),
    as functions of the radial coordinate $r$ and the axial coordinate $z$.
    %
    We use SI units.
    %
    The constant $\mu$ is the dynamic viscosity of the fluid
%     has the unit of [pressure $\times$ time]
    that expresses its resistance to flow.
    %
    The ``no-slip'' boundary conditions 
    dictate that 
    the velocity must be zero at the wall.
    
    \begin{itemize}
    \item  
        Why are both sides equal to some constant $C$?
        What is the physical meaning of this constant?
    \item
        Assume the pressure drop across the pipe 
        of length $L$ is $\Delta P < 0$.
        What is the flow rate [volume / time]
        of the fluid?
    \item
        Estimate the flow rate out of the water tap
        and 
        thus the pressure at the source.
    \end{itemize}


    %
    
    
    
    


    \vfill\hfill 
    {\small\textcopyright} R.A., \today
\end{document}
