\documentclass[12pt,a4paper]{article}

\usepackage[utf8]{inputenc}
\usepackage[english]{babel}
\usepackage{amssymb, amsmath}
\usepackage{fullpage}
\usepackage{parskip}
\usepackage{url}

\newcommand{\from}{\colon}
%
\newcounter{task}[section]
\newcommand{\newtask}{\refstepcounter{task}{\vspace{\baselineskip}\textbf{1.\thetask}}}
%
\newcommand{\norm}[1]{\left\|#1\right\|}

\begin{document}

    \begin{minipage}{0.6\textwidth}
        Differential equations
    \end{minipage}
    \hfill
    \begin{minipage}{0.4\textwidth}
        \raggedleft
        ELTE, autumn 2019/20
    \end{minipage}
    
    
    \newtask %%%%%%%%%%%%%%%%%%%%%%%%%%%%%%%%%%%%%%%%%%%%%%%%%%
    
    Sketch all possible continuously differentiable solutions
    of $\dot{x}(t) = -x(t)$, 
    $t \in \mathbb{R}$.
    %
    What do they have in common?

    
    \newtask %%%%%%%%%%%%%%%%%%%%%%%%%%%%%%%%%%%%%%%%%%%%%%%%%%
    
    Let $\chi$, $g$ and $a \geq 0$ be continuous real-valued
    functions on $[0, T]$, and assume that
    \begin{align}
        \chi(t) 
        \leq 
        g(t) +
        \int_0^t \chi(s) a(s) ds
        ,
        \quad
        \forall t \in [0, T]
        .
    \end{align}
    %
    \begin{itemize}
    \item 
        Prove the Gronwall inequality:
        %
        \begin{align}
            \chi(t) 
            \leq
            g(t) +
            \int_0^t 
                g(s)
                a(s)
                \exp \left( \int_s^t a(\tau) d\tau \right)
            ds,
            \quad
            \forall t \in [0, T].
        \end{align}
        
        Hint:
        start with a differential equation for 
        $x(t) := \int_0^t \chi(s) a(s) ds$.
    \item
        Assume $g$ is differentiable.
        %
        Prove the corollary:
        %
        \begin{align}
            \chi(t) 
            \leq 
            g(0) \int_0^t a(s) ds
            +
            \int_0^t g'(s) 
            \exp \left(
                \int_s^t a(\tau) d\tau
            \right)
            ds
            ,
            \quad
            \forall t \in [0, T].
        \end{align}
    \item
        Suppose $f \from [0, T] \times \mathbb{R} \to \mathbb{R}$
        is Lipschitz continuous in the second variable.
        %
        Let $x$ solve the differential equation 
        $\dot{x}(t) = f(t, x(t))$ on the interval $(0, T)$.
        %
        Show that the map
        $x(0) \mapsto x(T)$
        is Lipschitz continuous.
        %
        Is your statement sharp?
    \end{itemize}
    
    
    
    
    \newtask %%%%%%%%%%%%%%%%%%%%%%%%%%%%%%%%%%%%%%%%%%%%%%%%%%

    Develop a solution formula to
    the initial value problem
    \begin{align}
        \dot{x}(t) = g(t) h(x(t)),
        \qquad
        x(t_0) = x_0
        ,
    \end{align}
    in terms of the given functions $g$ and $h$.
    %
    Apply to
    \begin{itemize}
    \item
        $\dot{x}(t) = t \, x(t)$,
    \item
        $\dot{x}(t) = t / x(t)$,
    \item
        $\dot{x}(t) = x^2(t)$,
    \item  
        $\dot{x}(t) = \sqrt{ x(t) }$,
    \item
        $\dot{x}(t) = (1 + t^2) (1 + x^2(t))$,
    \item
        $\dot{x}(t) = x(t) \log x(t)$.
    \end{itemize}
    %
    What is the maximal temporal interval of existence in each case?

    
    \newtask %%%%%%%%%%%%%%%%%%%%%%%%%%%%%%%%%%%%%%%%%%%%%%%%%%
    
    Give the maximal interval of existence $(0, t^\star)$
    for the initial value problem
    $\dot{x}(t) = x^2(t)$
    with
    $x(0) := x_0 > 0$.
    
    \begin{itemize}
    \item 
        Why is $t^\star$ called the ``blow up time''?
    \item
        Can you continue the solution past the blow up time?
    \end{itemize}

    
    Now consider 
    $\dot{z}(t) = z^2(t)$, $t > 0$,
    for a complex-valued function $z$
    with the intial value
    $z(0) = x_0 + i \epsilon$,
    where $x_0 > 0$ and $\epsilon \neq 0$ are real.
    %
    Where needed, assume $x_0 = 1$.

    \begin{itemize}
    \item 
        What happens to the blow up time?
    \item
        Decomposing $z = x + i y$,
        write down the differential equations for $x$ and $y$.
    \item
        Argue that $x^2(t) + (y(t) - R_0)^2 = R_0^2$
        for a certain real constant $R_0$
        and
        comment.
    \item
        Sketch the evolution in $(t, x, y)$-space 
        for a small $\epsilon > 0$.
        %
        What happens for $\epsilon \searrow 0$?
    \end{itemize}

    
    
    \newtask %%%%%%%%%%%%%%%%%%%%%%%%%%%%%%%%%%%%%%%%%%%%%%%%%%
    
    Develop a solution formula to
    the initial value problem
    \begin{align}
        \dot{x}(t) = a(t) x(t) + b(t)
        ,
        \qquad
        x(t_0) = x_0
        ,
    \end{align}
    in terms of the given functions $a$ and $b$.
    %
    \begin{itemize}
    \item
        Suppose $a$ and $b$ are indenpendent of $t$.
        Investigate the limits $a \to 0$ and $b \to 0$.
    \item
        Apply your formula to $t \, \dot{x}(t) = 2 x(t) + t$.
        %
        Don't forget to discuss $t = 0$.
    \item
        Check your formula on examples of your choice.
    \item
        Is this a good model for global seafood stock?
    \item
        Via a suitable transformation,
        solve 
        $\dot{y}(t) = y(t) - y^\alpha(t)$,
        where $\alpha \in \mathbb{R}$.
        %
        What is remarkable about this 
        ``Bernoulli differential equation''?
    \item
        Solve the ``logistic differential equation''
        $\dot{y}(t) = y(t) (1 - y(t))$
        with $0 \leq y(0) \leq 1$.
        %
        Explain this model
        in terms of ``growth'' and ``competition''.
    \item
        Propose a better model for global seafood stock.
    \end{itemize}

    
    \newtask %%%%%%%%%%%%%%%%%%%%%%%%%%%%%%%%%%%%%%%%%%%%%%%%%%
    
    Propose a nontrivial system of differential equations
    ``$\dot{x}(t) = \ldots$,
    $\dot{y}(t) = \ldots$''
    together with an initial condition at $t = 0$
    for which the function
    %
    \begin{align}
        t \mapsto v(t) := x^2(t) + y^2(t)
    \end{align}
    %
    satisfies, for all $t > 0$,
    %
    \begin{itemize}
    \item
        $\dot{v}(t) = -2 v(t)$.
    \item 
        $\dot{v}(t) = 0$.
    \item
        $\dot{v}(t) = 2 x(t) + 2 y(t)$.
    \end{itemize}
    %
    In each case, 
    comment graphically on the possible trajectories $t \mapsto (x(t), y(t))$ in $\mathbb{R}^2$.

    

    \newtask %%%%%%%%%%%%%%%%%%%%%%%%%%%%%%%%%%%%%%%%%%%%%%%%%%
    
    On the interval $I := (0, 1)$
    define
    the functions
    $
        \varphi_k := \sqrt{2} \sin(\pi k \,\cdot\,)
    $.
    %
    Check whether the set
    $\Phi := \{ \varphi_k : k = 1, 2, 3, \ldots \}$
    is
    %
    \begin{itemize}
    \item
        orthonormal
        in the Lebesgue space $L_2(I)$,
    \item
        a basis thereof.
    \end{itemize}

    Now consider the heat equation
    \begin{align}
        \partial_t u(t, x) - \partial_{x x} u(t, x)
        & =
        f(t, x),
        \qquad
        t > 0,
        \quad
        x \in I
        ,
    \end{align}
    with the initial condition $u(0, \,\cdot\,) = g$
    and the boundary conditions $u|_{\partial I} = 0$.
    
    \begin{itemize}
    \item
        What is the physical motivation for this PDE?
    \item
        Solve the PDE formally by expansion in a suitable spatial basis.
    \item
        What qualitative properties of $u$ can you infer?
    \item
        Can you propose a suitable functional framework?
    \item
        Does the solution $u$ depend continuously 
        on the data $f$ and $g$?
    \end{itemize}

    
    \newtask %%%%%%%%%%%%%%%%%%%%%%%%%%%%%%%%%%%%%%%%%%%%%%%%%%
    
    These days, papers are published with a short abstract
    or even with a few ``highlights'' as bullet points. 
    %
    Write the ``highlights''
    emphasizing the role of differential equations for 
    \begin{quote}
        \textsc{R.~M.~Solow},
        \emph{A contribution to the theory of economic growth},
        The quarterly journal of economics, 70 (1956), pp.~65--94.
    \end{quote}
    according to Elsevier's guidelines: 
    \begin{quote}
        {\small\url{https://www.elsevier.com/authors/journal-authors/highlights}}
    \end{quote}
    
    


    \vfill
    \hfill
    R.A.,
    \today
\end{document}
